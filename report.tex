\documentclass[a4 paper, 12pt]{article}
\usepackage{hyperref}
\usepackage{float}
\usepackage[pdftex]{graphicx}
\begin{document}
\author{Neil McCaffrey}
\title{A Look At The History Of The Gimp And Its Impact On Open Source
Software}
\date{\today}
\maketitle

%Ok so what I want to say in this report:
%how the gimp started and what motivated its inseption.
%show how a comunity got built around this peice of software
%give some insite as to the development of the gimp during the mid 2000s. What
%choices were made that influenced the success/failure of the project. What
%projects have come from the development of the gimp? Is the gimp still the
%flagship opensource project that it always was?

%Introduce the thing
	There have been few open source applications that have received as much
	attetion in the media as the GIMP (GNU Image Manipulation Program). The
	project has been compared to large comercial software like Adobe Photoshop.
	It has spawned a large supporting community around it. Although these
	comparisons have been unfavorable I want to make the argument that the GIMP
	is more important than just being a free image editor.

	The GIMP started life as a college asignment for students Spencer Kimball
	and Peter Mattis in 1995. Rather than write a compiler in Scheme/Lisp they
	decided to write an image manipulation program. They worked on the GIMP for
	about 10 months before releasing a working program in January of 1996. This
	initial releas had features such as basic channel operation and a plug-in
	archetecture which allowed users to extend the functionality of the
	program.

	However the first release was far from perfect, because of its dependancy
	on the Motif toolkit for rendering GUI widgits distribution of the GIMP was
	initially difficult. The use of Motif also made the GIMP prone to crashing.
	Because of these problems Peter got fed up with the Motif toolkit and
	started work on the GIMP Tool Kit or GTK for the 0.6x series of the GIMP.
	%Need to fill in the blank drawn here, too bored right now.

	Wheather it was intended or not GTK+ became very popular due to the Gnome
	project. In 

	In June 1997 Peter and Spencer left the project, having graduated and gotten
	jobs they no longer had time to contribute to the project. This left the
	project in a state of confusion. Patches were contributed but not merged
	into the official relase. This lead to unofficial ``pre-releases'' to be
	made and published. To pick up the slack for this transition period Federico
	Mena Quintero took the lead of the project and lead it in the direction of
	stability and useability for the 1.0 release.

	With the founders of the project gone the development of the GIMP split up
	into teams developing the toolkit and a team developing the application.
	Where the project was to go next was discussed on  the \#gimp IRC channel.
	There was an obvious need for centeralisation of both the code and
	documentation so Shawn Amundson secured the domain gimp.org. The developers
	started to manage their code using CVS around this time.

	%Feels like there is a bit of a gap here too. :/
	Dispite its success in the opensource world comercial users were unsatisfied
	by the GIMP. This lead to a fork of the 1.0.4 version called CinePaint in
	1999.\cite{release} CinePaint is an aplication which has very different
	goals to the GIMP. It has a much higher colour depth in comparison to the
	GIMP, 32bit vs 8bit per colour channel. With high colour depth and advanced
	features such as onion skinning CinePaint is an aplication focused on film
	retouching. It has seen use in many films such as `Spider Man', `Lord of the
	Rings' and `Harry Potter and the Philosopher's Stone'. It is second only to
	Photoshop in the film industry for retouching still images from films.
	
	As a result of its success the GIMP has often been compared to Adobe's
	Photoshop. This left many new users disappointed when they realised it is a
	very different application. In responce forks of the GIMP have been made
	which attempt to replicate the Photoshop interface using the GIMP. GimpShop
	was first released in 2006 and was based on the 2.2.11 version of the GIMP.
	It featured menus which resembled those found in Photoshop and application
	windows were contained within a larger window like Photoshop. However
	development of the GimpShop stagnated soon after its release.

	Although having feature parity with Photoshop is not one of the GIMP
	developers' goals users have desired the application to have support for
	higher colour bit depth and CYMK colour channels. Rather than build this
	functionality directly into the application the developers decided to build
	an external library to handle these functions. This is the GEneric Graphics
	Library.(GEGL)

	Development of GEGL has been ongoing since 2000 when it was originally
	concieved as a replacement to GIMP core. It is modeled after an acyclic
	graph where each node represents an operation and each edge. A graph may be
	stored in an XML format and applied to multiple images for batch
	processing. The main advantage of representing edited images in this maner
	is non-destructive editing of a file.
	
\begin{thebibliography}{12}
	\bibitem{earlyHistory}
		\url{``http://www.gimp.org/about/ancient_history.html''}
	\bibitem{gnomeHistory}
		\url{``http://primates.ximian.com/~miguel/gnome-history.html''}
	\bibitem{release}
		\url{``http://www.gimp.org/about/history.html''}
	\bibitem{gnome}
		\url{``http://primates.ximian.com/~miguel/gnome-history.html''}
	\bibitem{gegl}
		\url{``http://gegl.org/''}
\end{thebibliography}
\end{document}
